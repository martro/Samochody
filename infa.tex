%%%%%%%%%%%%%%% Początek %%%%%%%%%%%%%%%
\documentclass[a4paper,12pt]{article}
\usepackage{polski}
\usepackage[T1]{fontenc}
\usepackage[utf8]{inputenc}
\usepackage{geometry}
\usepackage{url}
\usepackage {wrapfig}
\usepackage{listings}
\usepackage{color}
\usepackage{textcomp}
\definecolor{listinggray}{gray}{0.9}
\definecolor{lbcolor}{rgb}{0.9,0.9,0.9}
\lstset{
	backgroundcolor=\color{lbcolor},
	tabsize=4,
	rulecolor=,
	language=matlab,
        basicstyle=\scriptsize,
        upquote=true,
        aboveskip={1.5\baselineskip},
        columns=fixed,
        showstringspaces=false,
        extendedchars=true,
        breaklines=true,
        prebreak = \raisebox{0ex}[0ex][0ex]{\ensuremath{\hookleftarrow}},
        frame=single,
        showtabs=false,
        showspaces=false,
        showstringspaces=false,
        identifierstyle=\ttfamily,
        keywordstyle=\color[rgb]{0,0,1},
        commentstyle=\color[rgb]{0.133,0.545,0.133},
        stringstyle=\color[rgb]{0.627,0.126,0.941},
        language=C,
}

\title{Baza danych komisu samochodowego}
\author{Marcin Trojan}
\date{\today}

\begin{document}
\maketitle

\section{O programie}
Program to prosta aplikacja bazodanowa, napisana w języku C.

\subsection{Opis funkcji menu}

\begin{enumerate}
  \item \textbf{WYSWIETL} - Funkcja wyświetla wszystkie samochody znajdujące się w buforze. Gdy bufor jest pusty, informuje komunikatem "Lista jest pusta".
  
  \item \textbf{SORTUJ} - Sortowanie wg wszystkich parametrów samochodu (marka, model, cena, przebieg, typ paliwa, nowy/używany, bezwypadkowy/powypadkowy, rok produkcji). Aby zobaczyć efekt sortowania należy wyświetlić bufor.
  \item \textbf{SZUKAJ} - Wyświetlanie samochodów spełniających parametry wyszukiwania. Gdy wybrane zostanie wyszukiwanie wg marki lub modelu, można użyć znaków specjalnych '*' (dowolny ciąg znaków) i '?' (dowolny, pojedyńczy znak).
  \item \textbf{DODAJ NOWY SAMOCHOD} - Dodanie nowego samochodu do bazy na podstawie dialogu z uźytkownikiem.
  \item \textbf{EDYTUJ DANE SAMOCHODU} - 
  \item \textbf{USUN SAMOCHOD} - Usuwanie dowolnego samochodu znajdującego się w buforze.
  \item \textbf{DODAJ LOSOWY SAMOHOD} - Program generuje dane nowego samochodu. Parametry marka i model są generowane wg założenia, że słowo składa się z na przemian ułożonych spółgłosek i samogłosek. Długość słowa jest losowana. Pozostałe parametry spełniają założenia programu, jednak nie zawsze są logiczne (np. auto z 1960r. może nie mieć nic na liczniku :) ).
  \item \textbf{ZAPISZ BUFOR} - Zapisuje dane znajdujące się w buforze. Możliwy wybór, czy nazwa pliku ma być domyślna, ostatnio używana, czy nowa. Gdy plik o tej samej nazwie już istnieje, zostanie nadpisany.
  \item \textbf{WCZYTAJ BUFOR} - Wczytuje bazę danych do bufora. Gdy plik nie istnieje, nic nie zostanie wczytane i lista nie będzie zajmować pamięci.
  \item \textbf{USUN BUFOR BEZ ZAPISYWANIA} - Czyści pamięć RAM używaną przez program. Dane znajdujące się w pamięci są tracone.
  \item \textbf{ZAKONCZ} - Kończy działanie programu. Gdy w pamięci znajdują się dane, program prosi użytkownika o informację, czy powinny zostać zapisane, czy usunięte.

\end{enumerate} 



\end{document}
